\documentclass[margin, 10pt]{res} % Use the res.cls style, the font size can be changed to 11pt or 12pt here
\usepackage{helvet} % Default font is the helvetica postscript font
\usepackage{hyperref}
\setlength{\textwidth}{5.1in} % Text width of the document
\begin{document}

    %----------------------------------------------------------------------------------------
    %	NAME AND ADDRESS SECTION
    %----------------------------------------------------------------------------------------

    \moveleft.5\hoffset\centerline{\large\bf Salim Fadhley} % Your name at the top

    \moveleft\hoffset\vbox{\hrule width\resumewidth height 1pt}\smallskip % Horizontal line after name; adjust line thickness by changing the '1pt'

    \moveleft.5\hoffset\centerline{66 Nightingale Road} % Your address
    \moveleft.5\hoffset\centerline{London, N228PP}
    \moveleft.5\hoffset\centerline{+44 7973 710574}
    \moveleft.5\hoffset\centerline{salimfadhley@gmail.com}
    \moveleft.5\hoffset\centerline{https://github.com/salimfadhley}

    %----------------------------------------------------------------------------------------

    \begin{resume}

        %----------------------------------------------------------------------------------------
        %	OBJECTIVE SECTION
        %----------------------------------------------------------------------------------------

        \section{AVAILABILITY}

        Available for London-based contract work in January 2019.

        %----------------------------------------------------------------------------------------
        %	Technology SKILLS SECTION
        %----------------------------------------------------------------------------------------

        \section{TECHNOLOGY \\ SKILLS}

        {\sl Development:} Python, Pandas, Scala, JavaScript, Groovy, Java, Go, Protcol Buffers, \LaTeX, OpenAPI, Swagger \\
        {\sl DevOps:} AWS, Azure, Terraform, Jenkins, Ansible, OpenShift, Travis, Ansible, GitHub, BitBucket, Linux, SQL Databases, SBT, Docker\\

        %----------------------------------------------------------------------------------------
        %	PROFESSIONAL EXPERIENCE SECTION
        %----------------------------------------------------------------------------------------

        \section{WORK EXPERIENCE}

        {\sl Developer} - \textbf{Self Employed} \hfill Octomber 2018 - December 2018 \\

         I'm working on a private project until the end of this year.

        {\sl Software Engineer, Kernel team} - \textbf{Thought Machine} \hfill August 2018 - October 2018 \\

         I work on the ledger and payment execution service within the "kernel" of Thought Machine's Vault platform - a
         platform for core banking.

         I develop scalable micro-services in Python, Go and Java using technologies such as Protocol Buffers and GRPC.
         This is all deployed using Kubernetes on the Amazon and Google clouds.

        {\sl VP Software Engineer, Core DevOps team} - \textbf{Bank of America} \hfill January 2018 - August 2018 \\

        I help manage the Horizon software development tools platforms: This Atlassian Stack service layer
        brings together Atlassian Stack + Jenkns. I maintain the
        microservices associated with the platform, ensuring that everything is automatically built,
        tested and trivial to deploy.

        {\sl Contract Software Engineer, Global Mid-Office Technology} - \textbf{Bank of America} \hfill April 2012 - December 2017 \\

        \begin{itemize}
            \item I was a founding member of the QzRex team, which helped build the strategic reconciliation tool used by the Quartz Risk management platform.
            \item I worked on the Global Reconciliation System platform, building tools which help the deployment and release QA process. Later I worked on a number of projects which re-skinned this legacy application by giving it a Quartz user interface, and then later an Angular.JS user interface based on a high-performance Scala back end.
            \item Most recently I was a core-team member for the Insight Regulatory Reporting platform, helping to build the core-systems for the bank's real-time MIFID2 reporting engine.
            \item Outside of my core projects, I volunteered to build solutions that would make the global BAML developer community more effective. For example by creating documented, reusable pipelines for building, testing, releasing and deploying projects written in Python, JavaScript and Scala.
        \end{itemize}

        {\sl Quant Support Developer} - \textbf{Credit Agricole} \hfill June 2007 - April 2012 \\

        I was the primary Python developer of a multi-language commodities pricing library.

        \begin{itemize}
            \item Build and Support a number of excel-based pricing tools which used the Python library.
            \item Technical support for the quant research and validation teams to enable them to contribute correct, high-performance financial models
            \item Code quality \& performance of the entire library. Ensuring adequate test-coverage and correctness via a continuous integration system
        \end{itemize}

        {\sl Python Developer} - \textbf{Standard Bank} \hfill December 2006 - June 2007\\

        Python development on a Sunguard Front Arena based risk management system.

        My role was to implement custom reports and instrument-types.

        I also an a number of migration projects and developed a range of tools intended to automate and simplify the deployment process.

        {\sl Python Developer} - \textbf{Morgan Stanley} \hfill December 2005 - June 2006\\


        Building a data-warehouse of historical market data, to support the quant-team's research activities.

        I built a simple Python API on an assortment of proprietary SOAP, XML-RPC, Binary and plain-text interfaces commonly used within the organisation. I united a large number of inconsistently developed corporate standards behind a single simple to use Python API. This required an understanding of a range of structured credit products and concepts - where possible my Python layer corrected historic structural errors in the data's representation.

        I developed a number of abstract interfaces representing common credit risk concepts for use by model builders in the quant-team.

        {\sl Python Developer} - \textbf{Deskforce Ltd} \hfill March 2005 - April 2005\\

        \begin{itemize}
            \item Contributing Developer of Plone4Artists - A Plone based Podcasting tool.
            \item Development of ATAsterisk - An Archetypes based integration layer for the Asterisk PBX system.
        \end{itemize}

        {\sl Python Developer} - \textbf{Royal Bank of Scotland} \hfill November 2004 - April 2005\\

        Lead developer on a large Zope/Plone based intranet.

        {\sl Python Developer} \hfill Deskforce Ltd - March 2001 - November 2004\\

        {\sl Interactive Marketing Consultant} - \textbf{Ogilvy \& Mather} \hfill May 1998 - March 2001\\

        {\sl Graduate Developer} - \textbf{Good Technology} \hfill June 1997 - May 1998\\

        \section{EDUCATION}

        {\sl BSC, Computer Science:} University of Nottingham, 1994 - 1998. \\


        \section{EXTRA-CURRICULAR \\ ACTIVITIES}


        \begin{itemize}
            \item Improv Comedy with the Free Association
            \item Cycling
            \item Running
        \end{itemize}


        {\tiny This CV has been built with \LaTeX, compiled from source on \url{https://github.com/salimfadhley/cv/settings}{GitHub} in \url{https://travis-ci.org/salimfadhley/cv}{Travis-CI} and then
        automatically published. You can get the latest version of this document from \url{https://salimfadhley.github.io/cv/cv.pdf}{GitHub Pages}}

    \end{resume}
\end{document}